%%
% Please see https://bitbucket.org/rivanvx/beamer/wiki/Home for obtaining beamer.
%%
\documentclass{beamer}
\mode<presentation>
%\usetheme{Boadilla}
\usetheme{albany}
%\usecolortheme{dolphin}

% Packagess
\usepackage{graphicx}
\usepackage{subfigure}
%\usepackage{mwe}
\usepackage{capt-of}
\usepackage{tabu}
\usepackage{amsthm,amsmath,amssymb}
\usepackage{float}
\usepackage[bottom]{footmisc}
\usepackage[square,numbers,sort&compress]{natbib}
\usepackage{algorithm}
\usepackage{algorithmic}

%\usepackage[svgnames]{xcolor}
%\usepackage{caption}
%\captionsetup[figure]{labelfont={color=purple}}
%\captionsetup[table]{labelfont={color=purple}}
%\usepackage[font={color=purple,bf}]{caption}
%\makeatletter
%\def\BState{\State\hskip-\ALG@thistlm}
%\makeatother
\usepackage[utf8]{inputenc}
\usepackage[english]{babel}
\usepackage[autostyle, english = american]{csquotes}
\MakeOuterQuote{"}
 %\renewcommand{\figurename}{\color{purple}{figure}}
 \definecolor{purple}{RGB}{61,22,96}
\definecolor{gold}{RGB}{236,168,19}
 \makeatletter
 \def\th@mystyle{%
    \normalfont % body font
    \setbeamercolor{block title example}{bg=purple,fg=gold}
    \setbeamercolor{block body example}{bg=purple!20,fg=black}
    \def\inserttheoremblockenv{exampleblock}
  }
\makeatother
\theoremstyle{mystyle}
\setbeamertemplate{theorems}[numbered] 
\newtheorem{thm}{Theorem}
\newtheorem{defn}{Definition}
\newtheorem{code}{Algorithm}


\newcommand\blfootnote[1]{%
\begingroup
\renewcommand\thefootnote{}\footnote{#1}%
\addtocounter{footnote}{-1}%
\endgroup
}


\title[Presentation]{Presentation Slide Template}
\author[MNZY]{Michael Natole and Zhenhuan Yang}
\institute[UAlbany]{Department of Mathematics \& Statistics\\University at Albany, SUNY}
\date{July 12, 2022}

\begin{document}

% Constants
% \input{constants}


\begin{frame}
	\titlepage
\end{frame}


\begin{frame}
	\frametitle{Overview}
	\tableofcontents
\end{frame}

% First section
\section[First]{First Section}

\begin{frame}
	\frametitle{Outline}
	\tableofcontents[currentsection]
\end{frame}

\begin{frame}{Theorems}
\begin{thm}\label{thm:euler}
Euler's formula
\[
	e^{i\pi} + 1 = 0.
\]
\end{thm}
\begin{thm}\label{thm:einstein}
Mass–energy equivalence
\[
	E = mc^2
\]
\end{thm}
\end{frame}

\begin{frame}{Itemization}
\begin{itemize}
\item First item 
\item Second item
\item ...	
\end{itemize}
	
\end{frame}

\begin{frame}{Algorithm}
\begin{code}[Gradient Descent]
\begin{algorithmic}[1]
\STATE Choose an initial vector of parameters $w$ and learning rate $\eta$
\WHILE {True}
\STATE $w = w - \eta\nabla f(w)$
\ENDWHILE
\end{algorithmic}
\end{code}

\end{frame}


% Second section
\section[Second]{Second Section}

\begin{frame}
	\frametitle{Outline}
	\tableofcontents[currentsection]
\end{frame}

\begin{frame}{Citation}
Cite \citet{einstein1935can} using within the text format and then cite again with parenthetical citation \citep{kingma2014adam}.

\end{frame}

\begin{frame}[allowframebreaks]
	\frametitle{References}
	\bibliographystyle{plainnat}
	{\footnotesize
	\bibliography{bibfile}}
\end{frame}


\end{document}
